\section{Teaching Methods}
%%%%%%%%%%%%%%%%%%%%%%%%%%%%%%%%%%%%%%%%%%%%%%%%%%%%%%%%%%%
% Please modify the following text to best suit the course
%%%%%%%%%%%%%%%%%%%%%%%%%%%%%%%%%%%%%%%%%%%%%%%%%%%%%%%%%%%
The teaching methods for a Machine Learning course encompass a range of strategies that engage students with both theoretical concepts and practical applications. Here are the key teaching methods for such a course:
\begin{itemize}
    \item Lectures and Theoretical Foundations: Engage students through interactive lectures that introduce fundamental concepts in machine learning, data processing, neural networks, deep learning, and Data. Provide a solid theoretical foundation to build upon.

    \item Project-Based Learning: Assign projects that challenge students to design and implement Machine Learning solutions. Projects can include Model evaluation and selection and practical application.

    \item Discussion and Peer Learning: Facilitate class discussions where students can share their insights, ask questions, and discuss challenges they encounter while working on projects. Peer learning encourages collaboration and diverse viewpoints.

    %\item Case Studies: Analyze real-world case studies showcasing the impact of machine learning in various industries and highlight challenges and solutions.

    %\item Student Presentations: Have students present their project outcomes and findings to the class. This develops their communication skills and allows them to showcase their practical achievements.
\end{itemize}
